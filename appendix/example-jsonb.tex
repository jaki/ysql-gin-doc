\hypertarget{example-jsonb}{%
\subsection{Example: jsonb}\label{example-jsonb}}

Here is an example of using a jsonb GIN index.

\begin{sqlcode}
CREATE TABLE records (p SERIAL PRIMARY KEY, j jsonb);
INSERT INTO records (j) VALUES
  ('{"a": 1}'),
  ('{"b": {"c": "d"}}'),
  ('{"b": {"c": "e"}}'),
  ('{"b": [1, [2, 3], 4], "c": "f"}');
CREATE INDEX ON records USING GIN (j jsonb_ops);
SET enable_seqscan = OFF;
EXPLAIN SELECT * FROM records WHERE j @> '{"b": {}}'; -- this is index scan
\end{sqlcode}

Example of what can be done, all using the index:

\begin{sqlcode}
SELECT * FROM records WHERE j @> '{"b": {}}';
SELECT * FROM records WHERE j ? 'c';
SELECT * FROM records WHERE j ?| ARRAY['a', 'c'];
SELECT * FROM records WHERE j ?& ARRAY['c', 'b'];
SELECT * FROM records WHERE j @? '$.b[*] ? (@ > 3)';
SELECT * FROM records WHERE j @@ '$.b.c == "e"';
\end{sqlcode}

You can think of

\begin{sqlcode}
SELECT * FROM records WHERE j @? '$.b[*] ? (@ == 3)';
\end{sqlcode}

to be like

\begin{sqlcode}
SELECT * FROM records WHERE (
    j @@ '$.b[0] == 3' or
    j @@ '$.b[1] == 3' or
    j @@ '$.b[2] == 3');
\end{sqlcode}
