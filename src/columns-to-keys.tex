For each GIN opclass, I give examples of converting the column I'm indexing to
the index key(s) I'm storing.

\begin{center}
  \begin{tabular}{lll}
    \toprule
    opclass & column & keys \\
    \midrule
    \sqlinline{tsvector_ops}
        & \sqlinline{to_tsvector('simple', 'fo qu ba fo ba')}
        & \sqlinline{"fo"}, \sqlinline{"qu"}, \sqlinline{"ba"} \\
    \sqlinline{array_ops}
        & \sqlinline{ARRAY[1, 2, 3, 2]}
        & \sqlinline{1}, \sqlinline{2}, \sqlinline{3} \\
    \sqlinline{jsonb_ops}
        & \sqlinline{'{"a":"b", "c":{"d":[-1,[5.2]], "c":"b"}}'}
        & \sqlinline{"\001a"}, \sqlinline{"\005b"}, \sqlinline{"\001c"}, \\
      && \sqlinline{"\001d"}, \sqlinline{"\004-1"}, \sqlinline{"\0045.2"} \\
      & \sqlinline{'{"a":{}, "b":[]}'}
        & \sqlinline{"\001a"}, \sqlinline{"\001b"} \\
      & \sqlinline{'[20, 20.0, 20.000]'}
        & \sqlinline{"\00420"} \\
    \sqlinline{jsonb_path_ops}
        & \sqlinline{'{"a":"b", "c":{"d":[-1,[5.2]], "c":"b"}}'}
        & \sqlinline{2076393154}, \sqlinline{3631049813}, \\
      && \sqlinline{3671652104}, \sqlinline{3705026877} \\
      & \sqlinline{'{"a":{}, "b":[]}'}
        & (none) \\
      & \sqlinline{'[20, 20.0, 20.000]'}
        & \sqlinline{805562689} \\
    \sqlinline{jsonb_full_ops}
        & \sqlinline{'{"a":"b", "c":{"d":[-1,[5.2]], "c":"b"}}'}
        & \sqlinline{"\Ka\Sb"}, \\
      && \sqlinline{"\Kc\Kd\A\N-1"}, \\
      && \sqlinline{"\Kc\Kd\A\A\N5.2"}, \\
      && \sqlinline{"\Kc\Kc\Sb"} \\
      & \sqlinline{'{"a":{}, "b":[]}'}
        & \sqlinline{"\Ka\K"}, \sqlinline{"\Kb\A"} \\
      & \sqlinline{'[20, 20.0, 20.000]'}
        & \sqlinline{"\A\N20"} \\
    \bottomrule
  \end{tabular}
\end{center}

To interpret \sqlinline{"\001"} characters, see gin flags in
\protect\hyperlink{%
  constants}{%
  the constants section}.  \sqlinline{"\A"} characters are similar, but they
map to bytes not found in any valid UTF-8 string so that there are no
ambiguities.  For example, if \sqlinline{"\K"} mapped to \sqlinline{"\001"}, it
would be ambiguous whether, upon seeing \sqlinline{"\001"}, a key ends or
continues with a literal \sqlinline{"\001"} character.  Since numbers are
always values, there are no ambiguity issues with them, thankfully.
\sqlinline{"\A"} represents an array nest level; \sqlinline{"\K"}, a key;
\sqlinline{"\N"}, a number; \sqlinline{"\S"}, a string.
