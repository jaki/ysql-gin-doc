Scan entries are formed in \cinline{ginNewScanKey} called by
\cinline{gingetbitmap}.  Then, they are categorized to required and additional
in \cinline{startScan}.  Put a breakpoint after that line:

\begin{textcode}
b ginget.c:1953
\end{textcode}

Run a select that exercises the index.

\begin{sqlcode}
CREATE TABLE tstab (tsv tsvector);
CREATE INDEX ON tstab USING gin (tsv);
SELECT * FROM tstab WHERE tsv @@ to_tsquery('abc');
\end{sqlcode}

When the breakpoint hits, you'll have \cinline{IndexScanDesc scan} loaded with
the scan entries.  Some key things to observe are

\begin{textcode}
p $i = $j = $k = 0
p ((GinScanOpaque)scan->opaque)->nkeys
p ((GinScanOpaque)scan->opaque)->keys[$i].nuserentries
p ((GinScanOpaque)scan->opaque)->keys[$i].nrequired
p ((GinScanOpaque)scan->opaque)->keys[$i].nadditional
p *(text*)((GinScanOpaque)scan->opaque)->keys[$i].requiredEntries[$j++]->queryKey
p *(text*)((GinScanOpaque)scan->opaque)->keys[$i].additionalEntries[$k++]->queryKey
\end{textcode}

where \textinline{$i}, \textinline{$j}, and \textinline{$k} should vary
accordingly.  In this case, there should be one key and one required entry
whose key is \textinline{abc}.

The same can be done for other opclasses:

\begin{sqlcode}
CREATE TABLE jbtab (jb jsonb);
CREATE INDEX ON jbtab USING gin (jb);
SELECT * FROM jbtab WHERE jb @> '{"def":"ghi"}';
\end{sqlcode}

In this case, there should be one key and two entries (one required; one
additional) whose keys are \textinline{\001def} and \textinline{\005ghi}.  The
first byte contains flags that, in this case, tell that the type is key and
string, respectively.

If we change the opclass, the scan entries should change.

\begin{sqlcode}
DROP INDEX jbtab_jb_idx;
CREATE INDEX ON jbtab USING gin (jb jsonb_path_ops);
SELECT * FROM jbtab WHERE jb @> '{"def":"ghi"}';
\end{sqlcode}

In this case, there should be one key and one entry (required) whose key is
formatted as an unsigned 32-bit integer.  Therefore, don't cast as
\cinline{text*} this time:

\begin{textcode}
p ((GinScanOpaque)scan->opaque)->keys[0].requiredEntries[0]->queryKey
\end{textcode}

You should get \cinline{903080546}.  This is can be derived as follows:

\begin{textcode}
p $def = hash_bytes("def", 3)
p $def_rotate = ($def << 1 ) | ($def >> 31)
p $ghi = hash_bytes("ghi", 3)
p $defghi = $def_rotate ^ $ghi
\end{textcode}

See \cinline{JsonbHashScalarValue} for how this derivation is done.

It's also possible to go through the entries extracted from the index.  Create
two breakpoints, one for entries matching the current item and one for when the
current item succeeds the consistent function.

\begin{textcode}
ginget.c:1228
ginget.c:1238
\end{textcode}

For the first breakpoint, get the current item pointer and entry key (assuming
text type) using

\begin{textcode}
p entry->curItem
p *(text*)entry->queryKey
\end{textcode}

The second breakpoint just tells whether the recent item completely matched.
For example, with

\begin{sqlcode}
-- Recreate index without fastupdate.
DROP INDEX tstab_tsv_idx;
CREATE INDEX ON tstab USING gin (tsv) WITH (fastupdate = false);

INSERT INTO tstab VALUES
  ('aaa ddd'),
  ('bbb ddd'),
  ('ccc ddd');
SELECT * FROM tstab WHERE tsv @@ '(aaa & bbb) | (ccc & ddd)';
\end{sqlcode}

you get one key, three required entries \textinline{bbb}, \textinline{ccc},
\textinline{ddd}, and one additional entry \textinline{aaa}.  The first
considered item is \sqlinline{'bbb ddd'} since that's the min item in the
posting lists of \textinline{bbb}, \textinline{ccc}, and \textinline{ddd}.  The
consistent function is given the query and \textinline{aaa=false},
\textinline{bbb=true}, \textinline{ccc=false}, \textinline{ddd=true}, and it
fails.  The second considered item is \sqlinline{'ccc ddd'}, and it succeeds
(hits the second breakpoint).
