Here are the opclasses that can be used with GIN\footnote{%
  \url{https://www.postgresql.org/docs/13/gin-builtin-opclasses.html}%
  }:

\begin{center}
  \begin{tabular}{llll}
    \toprule
    opclass & in type & key type & supported operators \\
    \midrule
    \sqlinline{tsvector_ops}
        & \sqlinline{tsvector}
        & \sqlinline{text}
        & \sqlinline{@@}, \sqlinline{@@@} \\
    \sqlinline{array_ops}
        & \sqlinline{anyarray}
        & \sqlinline{anyelement}
        & \sqlinline{&&}, \sqlinline{<@}, \sqlinline{=}, \sqlinline{@>} \\
    \sqlinline{jsonb_ops}
        & \sqlinline{jsonb}
        & \sqlinline{text}
        & \sqlinline{?}, \sqlinline{?&}, \sqlinline{?|}, \sqlinline{@>},
          \sqlinline{@?}, \sqlinline{@@} \\
    \sqlinline{jsonb_path_ops}
        & \sqlinline{jsonb}
        & \sqlinline{int4}
        & \sqlinline{@>}, \sqlinline{@?}, \sqlinline{@@} \\
    \sqlinline{jsonb_full_ops}
        & \sqlinline{jsonb}
        & \sqlinline{text}
        & \sqlinline{?}, \sqlinline{?&}, \sqlinline{?|}, \sqlinline{@>},
          \sqlinline{@?}, \sqlinline{@@} \\
    \bottomrule
  \end{tabular}
\end{center}

Notice that \sqlinline{jsonb} has three opclasses.  The first two are in
upstream postgres; the third is inspired by CockroachDB's inverted index
RFC\footnote{%
  \url{https://github.com/cockroachdb/cockroach/blob/master/docs/RFCS/20171020_inverted_indexes.md}%
  }.

Opclasses are per-column, so you can have multiple on a single index:

See \protect\hyperlink{%
  columns-to-keys}{%
  the appendix} for examples of translating columns (in type) to keys (key
  type).

\begin{sqlcode}
CREATE INDEX ON bar USING gin (jsonb_col jsonb_ops, jsonb_col jsonb_path_ops);
\end{sqlcode}
