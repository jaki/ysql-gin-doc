When running queries that have jsonpath in the condition, a
\cinline{JsonPathGinNode} is internally formed.  You can see it by setting a
breakpoint after extracting it:

\begin{textcode}
b jsonb_gin.c:775
\end{textcode}

Run some query with jsonpath.

\begin{sqlcode}
CREATE TABLE jpdemo (jb jsonb);
CREATE INDEX ON jpdemo USING gin (jb);
SELECT * FROM jpdemo WHERE jb @@ 'strict $.abc == "foo"';
\end{sqlcode}

Generally, you want to look at

\begin{textcode}
p $i = 0
p node->type
p node->val.nargs
p *((text*)node->args[$i]->val.entryDatum)
\end{textcode}

where \textinline{$i} should vary accordingly.  In this case, there's a top
node of type \cinline{JSP_GIN_AND} with two arguments, both of type
\cinline{JSP_GIN_ENTRY}.  The first has value \textinline{\001abc}, and the
second has value \textinline{\005foo}.

When doing a lax query, notice a change in the structure.  The second argument
becomes a \cinline{JSP_GIN_OR} with two arguments \textinline{\001foo} and
\textinline{\005foo}.  This is to handle arrays, illustrated as follows:

\begin{sqlcode}
INSERT INTO jpdemo VALUES ('{"abc": ["bar", "foo"]}');
SELECT * FROM jpdemo WHERE jb @@ 'lax $.abc == "foo"';
\end{sqlcode}
