All the operators boil down to a combination of these primitives:

\begin{oparts}
\item
  $\opget_I(f)$:
  given index $I$ and predicate $f$, return set of tuples
  \[
    T = \braces{(i, p) \in I : f(i)}
  \]
  $f$ should be a simple predicate like
  \begin{oparts}
  \item
    $i < c$
  \item
    $i = c$
  \item
    $i > c$
  \item
    $i \geq c$
  \item
    $i \leq c$
  \item
    $i~\textnormal{starts with}~c$
  \end{oparts}
\item
  $\opand(T_1, T_2)$:
  given sets of tuples $T_1$, $T_2$, return
  \[
    T = \braces{(i, p) \in T_1 \cup T_2 : p \in K}
  \]
  where
  \[
    K = T_1[\pk] \cap T_2[\pk]
  \]
\item
  $\opor(T_1, T_2)$:
  given sets of tuples $T_1$, $T_2$, return
  \[
    T = T_1 \cup T_2
  \]
\item
  recheck: a catchall for additional operations
\end{oparts}

For example, the \sqlinline{=} operator for arrays makes sure that the indexed
array column equals the query array. This can be done by

\begin{nparts}
\item
  For each distinct element $e$ in the query array $A_q$, get a set of tuples
  $T_e$ where the tuple's row's array contains the element $e$:
  \[
    \forall e \in A_q, T_e = \opget_I(f)
  \]
  where
  \[
    f(i) = \parens{i \qeq e}
  \]
\item
  Intersect the tuples $T_e$ to filter out any tuples whose row's array doesn't
  contain \textbf{all} of the query array's elements:
  \[
    T = \opand_{e \in A_q}(T_e)
  \]
\item
  Recheck each row to make sure its array has the same number of (distinct)
  elements as the query array.
\end{nparts}

See translations of all operators to primitives in \protect\hyperlink{%
  operators-to-primitives}{%
  the appendix}.
