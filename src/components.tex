First, it helps to know the following:

\begin{oparts}
\item
access method:\footnote{%
  \url{https://www.postgresql.org/docs/13/catalog-pg-am.html}%
  } storage interface (e.g. \sqlinline{btree}, \sqlinline{lsm}, \sqlinline{gin})
\item
operator:\footnote{%
  \url{https://www.postgresql.org/docs/13/catalog-pg-operator.html}%
  } (e.g. \sqlinline{int4 >= int8} $\rightarrow$ \sqlinline{bool}, \sqlinline{-
int8} $\rightarrow$ \sqlinline{int8})
\item
operator family:\footnote{%
  \url{https://www.postgresql.org/docs/13/catalog-pg-opfamily.html}%
  } collection of operators + access method
\item
operator class:\footnote{%
  \url{https://www.postgresql.org/docs/13/catalog-pg-opclass.html}%
  } operator family + in type + key type
\end{oparts}

We only need to concern ourselves with things related to the GIN access method.
The components to cover are

\begin{nparts}
\item
operators
\item
operator classes
\item
primitives
\item
translations of operators to primitives
\end{nparts}
