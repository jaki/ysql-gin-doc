You can represent each GIN operator using the primitives:

tsvector operators to primitives:

\begin{center}
  \begin{tabular}{lll}
    \toprule
    signature & translation & notes \\
    \midrule
    \sqlinline{tsvector @@ tsquery} \\
    $\hookrightarrow$ \sqlinline{tsquery}: \sqlinline{|}
        & Given \sqlinline{col @@ 'foo | bar'}, \\
      & $T_l = \opget_I(i \qeq \mathtt{foo})$ \\
      & $T_r = \opget_I(i \qeq \mathtt{bar})$ \\
      & $T = \opor(T_l, T_r)$ \\
    $\hookrightarrow$ \sqlinline{tsquery}: \sqlinline{&}
        & Given \sqlinline{col @@ 'foo & bar'}, \\
      & $T_l = \opget_I(i \qeq \mathtt{foo})$ \\
      & $T_r = \opget_I(i \qeq \mathtt{bar})$ \\
      & $T = \opand(T_l, T_r)$ \\
    $\hookrightarrow$ \sqlinline{tsquery}: \sqlinline{!}
        & Given \sqlinline{col @@ '!foo'},
        & expensive; \\
      & $T = \opget_I(i \qeq \mathtt{foo})$
        & mem $\propto$ prevalence of RHS \\
      & seqscan tuples where pkey $p \not\in T[\pk]$ \\
    $\hookrightarrow$ \sqlinline{tsquery}: \sqlinline{:*}
        & Given \sqlinline{col @@ 'foo:*'}, \\
      & $T = \opget_I(i \qsw \mathtt{foo})$
        & need range-partition \\
    \bottomrule
  \end{tabular}
\end{center}

anyarray operators to primitives:

\begin{center}
  \begin{tabular}{lll}
    \toprule
    signature & translation & notes \\
    \midrule
    \sqlinline{anyarray && anyarray}
        & Given \sqlinline{col && ARRAY[3, 7]}, \\
      & $T_3 = \opget_I(i \qeq \mathtt{3})$ \\
      & $T_7 = \opget_I(i \qeq \mathtt{7})$ \\
      & $T = \opor(T_3, T_7)$ \\
    \sqlinline{anyarray <@ anyarray}
        & Given \sqlinline{col <@ ARRAY[3, 7]},
        & likely cheap: \\
      & $T_3 = \opget_I(i \qeq \mathtt{3})$
        & cost $\propto$ size of RHS \\
      & $T_7 = \opget_I(i \qeq \mathtt{7})$ \\
      & $T = \opor(T_3, T_7)$ \\
      & seqscan tuples where pkey $p \in T[\pk]$ \\
      & and recheck the operator \\
    \sqlinline{anyarray = anyarray}
        & Given \sqlinline{col = ARRAY[3, 7]},
        & likely cheap; \\
      & $T_3 = \opget_I(i \qeq \mathtt{3})$
        & mem $\propto$ size of RHS \\
      & $T_7 = \opget_I(i \qeq \mathtt{7})$ \\
      & $T = \opand(T_3, T_7)$ \\
      & seqscan tuples where pkey $p \in T[\pk]$ \\
      & and recheck the operator \\
    \sqlinline{anyarray @> anyarray}
        & Given \sqlinline{col @> ARRAY[3, 7]}, \\
      & $T_3 = \opget_I(i \qeq \mathtt{3})$ \\
      & $T_7 = \opget_I(i \qeq \mathtt{7})$ \\
      & $T = \opand(T_3, T_7)$ \\
    \bottomrule
  \end{tabular}
\end{center}

jsonb operators to primitives:

\begin{center}
  \begin{tabular}{lll}
    \toprule
    signature & translation & notes \\
    \midrule
    \sqlinline{jsonb ? text}
        & Given \sqlinline{col ? 'foo'}, \\
      & $T = \opget_I(i \qsw \mathtt{Kfoo})$ \\
    \sqlinline{jsonb ?& text[]}
        & Given \sqlinline{col ?& ARRAY['foo', 'bar']}, \\
      & $T_l = \opget_I(i \qsw \mathtt{Kfoo})$ \\
      & $T_r = \opget_I(i \qsw \mathtt{Kbar})$ \\
      & $T = \opand(T_l, T_r)$ \\
    \sqlinline{jsonb ?| text[]}
        & Given \sqlinline{col ?| ARRAY['foo', 'bar']}, \\
      & $T_l = \opget_I(i \qsw \mathtt{Kfoo})$ \\
      & $T_r = \opget_I(i \qsw \mathtt{Kbar})$ \\
      & $T = \opor(T_l, T_r)$ \\
    \sqlinline{jsonb @> jsonb}
        & Given \sqlinline{col @> '{"a":3, "b":{"c":["d"]}}'}, \\
      & $T_1 = \opget_I(i \qeq \mathtt{KaN3})$ \\
      & $T_2 = \opget_I(i \qeq \mathtt{KbKcASd})$ \\
      & $T = \opor(T_1, T_2)$ \\
    \sqlinline{jsonb @? jsonpath} \\
    \sqlinline{jsonb @@ jsonpath} \\
    $\hookrightarrow$ \sqlinline{jsonpath}: \sqlinline{==}
        & Given \sqlinline{col @@ '$.foo == 7'}, \\
      & $T = \opget_I(i \qeq \mathtt{KfooN7})$ \\
    $\hookrightarrow$ \sqlinline{jsonpath}: \sqlinline{>}
        & Given \sqlinline{col @@ '$.foo > 7'},
        & need simple pushdown; \\
      & $T = \opget_I((i \qsw \mathtt{KfooN}) \wedge
          (\textnormal{val}(i) \overset{?}{>} 7))$
        & need range-partition \\
    $\hookrightarrow$ \sqlinline{jsonpath}: \sqlinline{!=}
        & Given \sqlinline{col @@ '$.foo != 7'},
        & need range-partition \\
      & $T = \opget_I(i \qsw \mathtt{KfooN})$ \\
      & seqscan tuples where pkey $p \in T[\pk]$ \\
      & and recheck the operator \\
    $\hookrightarrow$ \sqlinline{jsonpath}: \sqlinline{!}
        & Given \sqlinline{col @@ '! ($.foo == 7)'},
        & expensive; \\
      & $T = \opget_I(i \qeq \mathtt{KfooN7})$
        & mem $\propto$ prevalence of RHS \\
      & seqscan tuples where pkey $p \not\in T[\pk]$ \\
    $\hookrightarrow$ \sqlinline{jsonpath}: \sqlinline{like_regex}
        & Given \sqlinline{col @@ '$.foo like_regex "bar$"'},
        & need range-partition \\
      & $T = \opget_I(i \qsw \mathtt{KfooS})$ \\
      & seqscan tuples where pkey $p \in T[\pk]$ \\
      & and recheck the operator \\
    $\hookrightarrow$ \sqlinline{jsonpath}: \sqlinline{starts with}
        & Given \sqlinline{col @@ '$.foo starts with "b"'},
        & need range-partition \\
      & $T = \opget_I(i \qsw \mathtt{KfooSb})$ \\
    $\hookrightarrow$ \sqlinline{jsonpath}: \sqlinline{ceiling}
        & Given \sqlinline{col @@ '$.foo.ceiling() == 5'},
        & need simple pushdown; \\
      & $T = \opget_I((i \qsw \mathtt{KfooN}) \wedge
          (\ceils{\textnormal{val}(i)} \qeq 5))$
        & need range-partition \\
    $\hookrightarrow$ \sqlinline{jsonpath}: \sqlinline{type}
        & Given \sqlinline{col @@ '$.foo.type() == "number"'},
        & need range-partition \\
      & $T = \opget_I(i \qsw \mathtt{KfooN})$ \\
    \bottomrule
  \end{tabular}
\end{center}
