The GIN key is of type \sqlinline{text}.  We can encode them to DocDB
\cppinline{kString}.  For example, \sqlinline{"the"} is encoded to
\cppinline{"Sthe\x00\x00"} for ascending and
\cppinline{"a\x8b\x97\x9a\xff\xff"} for descending.\footnote{see
\cppinline{AppendEncodedStrToKey}}  Since UTF-8 strings are guaranteed to not
have bytes \cppinline{"\x00"}, \cppinline{"\xfe"}, or \cppinline{"\xff"}, it
should be simple to create bounds for prefix search.  In fact, this is already
happening for queries like \sqlinline{col LIKE 'foo\%'}, which turn into a
\cinline{QL_OP_BETWEEN} of \cinline{"foo"} and \cinline{"fop"}.

Range partition is needed for efficient prefix queries.  Otherwise, hash
partition is fine.

Example:

\begin{sqlcode}
CREATE TABLE tsvtab (i int, ts1 tsvector, ts2 tsvector, PRIMARY KEY (i ASC));
CREATE INDEX ON tsvtab USING gin (ts1 ASC, ts2 DESC);
INSERT INTO tsvtab VALUES (4, 'abc abc', 'def ghi');
\end{sqlcode}

The index should contain

\begin{oparts}
\item
  \textinline{Sabc\x00\x00}, \textinline{a\x9b\x9a\x99\xff\xff},
  \textinline{SH\x80\x00\x01\x00\x01\x04!\x00\x00}, \textinline{!},
  \textinline{J\x80}, \textinline{#...}
\item
  \textinline{Sabc\x00\x00}, \textinline{a\x98\x97\x96\xff\xff},
  \textinline{SH\x80\x00\x01\x00\x01\x04!\x00\x00}, \textinline{!},
  \textinline{J\x80}, \textinline{#...}
\end{oparts}

\begin{sqlcode}
SELECT * FROM tsvtab WHERE ts1 @@ 'ab:*';
\end{sqlcode}

should look for

\begin{oparts}
\item
  $\geq$ \textinline{Sab\x00\x00}
\item
  $<$ \textinline{Sac\x00\x00}
\end{oparts}

\begin{sqlcode}
SELECT * FROM tsvtab WHERE ts2 @@ 'de:*';
\end{sqlcode}

should look for (TODO: double-check this)

\begin{oparts}
\item
  $>$ \textinline{a\x9b\x99\xff\xff}
\item
  $\leq$ \textinline{a\x9b\x9a\xff\xff}
\end{oparts}

TODO: look into \textinline{tsvector} \textbf{weights}
