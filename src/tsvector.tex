The GIN key is of type \sqlinline{text}.  We can encode them to DocDB
\cppinline{kString}.  For example, \sqlinline{"the"} is encoded to
\cppinline{"the\x00\x00"} for ascending and \cppinline{"\x8b\x97\x9a\xff\xff"}
for descending.  Since UTF-8 strings are guaranteed to not have bytes
\cppinline{"\x00"}, \cppinline{"\xfe"}, or \cppinline{"\xff"}, it should be
simple to create bounds for prefix search.  In fact, this is already happening
for queries like \sqlinline{col LIKE 'foo\%'}, which turn into a
\cinline{QL_OP_BETWEEN} of \cinline{"foo"} and \cinline{"fop"}.

Range partition is needed for efficient prefix queries.  Otherwise, hash
partition is fine.

TODO: look into \textinline{tsvector} \textbf{weights}
