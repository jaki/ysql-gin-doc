For postgres, \sqlinline{SELECT * FROM table_with_gin_index WHERE tscol @@
to_tsquery('simple', 'the')} does

\begin{nparts}
\item
  create scan key: \textinline{the}
\item
  get tuples matching scan key
\item
  recheck each tuple if needed
\end{nparts}

For Yugabyte, we should

\begin{nparts}
\item
  create scan key: \textinline{the}
\item
  \textbf{fetch tuples from DocDB matching scan key}
\item
  recheck each tuple if needed
\end{nparts}

This is a simplification as there are many more details, like multiple scan
keys, required/additional scan entries, and special search modes.  The details
for upstream postgres are in \protect\hyperlink{%
  read-path-extended}{%
  the appendix}.  I'll briefly outline what I believe should be the Yugabyte
approach:

\begin{nparts}
\item
  Support only one required scan entry, scanning all items of that entry and
  performing recheck.
\item
  Support multiple required scan entries with recheck.
\item
  Use consistent function and parallel streams of scan entries to further
  filter out items before fetching from the base table.  Since this may be less
  performant if it enables scans of the additional entries to avoid MAYBEs in
  the consistent function, consider making this a toggleable option rather than
  a hard upgrade.
\end{nparts}
