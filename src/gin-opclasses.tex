Indexes have an access method (e.g. \sqlinline{btree}, \sqlinline{lsm},
\sqlinline{gin}) and an opclass (e.g. \sqlinline{char_ops},
\sqlinline{numeric_ops}). \textbf{The opclass determines the index key format!}
Here are the opclasses that can be used with GIN (\href{%
  https://www.postgresql.org/docs/current/gin-builtin-opclasses.html}{%
  source}):

\begin{center}
    \begin{tabular}{lll}
        \toprule
        opclass & type & supported operators \\
        \midrule
        \sqlinline{tsvector_ops} & \sqlinline{tsvector} & \sqlinline{@@},
        \sqlinline{@@@} \\
        \sqlinline{array_ops} & \sqlinline{anyarray} & \sqlinline{&&},
        \sqlinline{@}, \sqlinline{=}, \sqlinline{@>} \\
        \sqlinline{jsonb_ops} & \sqlinline{jsonb} & \sqlinline{?},
        \sqlinline{?&}, \sqlinline{?|}, \sqlinline{@>}, \sqlinline{@?},
        \sqlinline{@@} \\
        \sqlinline{jsonb_path_ops} & \sqlinline{jsonb} & \sqlinline{@>},
        \sqlinline{@?}, \sqlinline{@@} \\
        \bottomrule
    \end{tabular}
\end{center}

If I \sqlinline{CREATE INDEX ON bar USING gin (jsonb_col)}, I implicitly use
opclass \sqlinline{jsonb_ops}. To use \sqlinline{jsonb_path_ops}, I need to
\sqlinline{CREATE INDEX ON bar USING gin (jsonb_col jsonb_path_ops)}.

\sqlinline{jsonb_ops} forms an index key for each individual piece (key or
value) of the jsonb.

\sqlinline{jsonb_path_ops} forms an index key for each path from root to leaf.

We could create our own opclass to form the keys in a way that fits with DocDB,
or we could instead have a translation layer that takes these postgres-derived
keys and converts them to DocDB format and vice versa.
