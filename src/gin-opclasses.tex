Indexes have an access method (e.g. \sqlinline{btree}, \sqlinline{lsm},
\sqlinline{gin}) and an opclass (e.g. \sqlinline{char_ops},
\sqlinline{numeric_ops}). \textbf{The opclass determines the index key format!}
Here are the opclasses that can be used with GIN (\href{%
  https://www.postgresql.org/docs/current/gin-builtin-opclasses.html}{%
  source}):

\begin{center}
  \begin{tabular}{lll}
    \toprule
    opclass & type & supported operators \\
    \midrule
    \sqlinline{tsvector_ops}
        & \sqlinline{tsvector}
        & \sqlinline{@@}, \sqlinline{@@@} \\
    \sqlinline{array_ops}
        & \sqlinline{anyarray}
        & \sqlinline{&&}, \sqlinline{<@}, \sqlinline{=}, \sqlinline{@>} \\
    \sqlinline{jsonb_ops}
        & \sqlinline{jsonb}
        & \sqlinline{?}, \sqlinline{?&}, \sqlinline{?|}, \sqlinline{@>},
          \sqlinline{@?}, \sqlinline{@@} \\
    \sqlinline{jsonb_path_ops}
        & \sqlinline{jsonb}
        & \sqlinline{@>}, \sqlinline{@?}, \sqlinline{@@} \\
    \bottomrule
  \end{tabular}
\end{center}

Notice that \sqlinline{jsonb} has two opclasses.

\begin{sqlcode}
CREATE INDEX ON bar USING gin (jsonb_col);
\end{sqlcode}

implicitly uses opclass \sqlinline{jsonb_ops}. To use
\sqlinline{jsonb_path_ops}, it must be explicitly stated:

\begin{sqlcode}
CREATE INDEX ON bar USING gin (jsonb_col jsonb_path_ops);
\end{sqlcode}
