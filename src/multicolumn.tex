If there's more than one column in the GIN index, it should also be part of the
record:

\begin{textcode*}{frame=none,xleftmargin=10pt}
[<gin_column_attnum>, <gin_index_key>, <primary_keys>]
\end{textcode*}

If the columns of the GIN index are of different key type, this will likely
pose a problem for DocDB since DocDB tables have fixed schemas.  Here are
several solutions:

\begin{oparts}
\item
  \emph{Create a DocDB table for each column.}  Problem is that, in postgres,
  the index will show up once, but it somehow maps to multiple indexes in
  DocDB.  It will probably take a lot of work to reconcile.
\item
  \emph{Have each GIN index column appear in the record, but make sure only one
  is active while the rest are null.}  Problem is that it's a waste of space,
  and the code needs to be careful not to violate the constraint.
\item
  \emph{Force all GIN index columns to map to text.}  Problem is that you're
  still forced to do all ASC or all DESC.  Also, some extra processing needs to
  be done to translate between types.
\item
  \emph{Relax the requirement for columns to have a fixed schema.}  This may
  also take work.
\end{oparts}

Since the attnum DocDB column is internal, allowing the user to specify hash or
range on the hidden attnum column will require changes up to the syntax.  For
now, it can be forced to be range partitioned.
