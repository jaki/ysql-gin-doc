\begin{sqlcode}
CREATE TABLE book (page int PRIMARY KEY, word text, position int);
\end{sqlcode}

creates a DocDB table with key \sqlinline{page HASH}. I can easily ask for
pages 5-12 using this primary key:

\begin{sqlcode}
SELECT * FROM book WHERE page >= 5 and page <= 12;
\end{sqlcode}

\begin{sqlcode}
CREATE INDEX ON book (word);
\end{sqlcode}

creates a secondary DocDB table with key \textinline{word HASH, page ASC}. This
is like the index at the back of a book. I can easily ask what pages have the
word \textinline{foo} using this index:

\begin{sqlcode}
SELECT page FROM book WHERE word = 'foo';
\end{sqlcode}

What if the table were structured instead like

\begin{sqlcode}
CREATE TABLE book (page int PRIMARY KEY, words text[]);
\end{sqlcode}

Now, looking for the specific word \textinline{foo} is time-consuming:

\begin{sqlcode}
select * from book where words && ARRAY['foo'];
\end{sqlcode}

Creating a regular index won't help since you still need to search
\textinline{words} for \textinline{foo}.
