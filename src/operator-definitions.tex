See almost all the operator definitions in PostgreSQL docs:

\begin{oparts}
\item
  \href{%
    https://www.postgresql.org/docs/12/functions-textsearch.html}{%
    tsvector operators}
\item
  \href{%
    https://www.postgresql.org/docs/12/functions-array.html}{%
    anyarray operators}
\item
  \href{%
    https://www.postgresql.org/docs/12/functions-json.html}{%
    jsonb operators}
\end{oparts}

\sqlinline{tsvector} operators:

\begin{center}
  \begin{tabular}{lll}
    \toprule
    signature & description & example \\
    \midrule
    \sqlinline{tsvector @@ tsquery}
        & match query
        & (see below) \\
    \bottomrule
  \end{tabular}
\end{center}

\sqlinline{tsquery} expression operators:

\begin{center}
  \begin{tabular}{lll}
    \toprule
    operator & description & example \\
    \sqlinline{expr | expr}
        & boolean OR
        & \sqlinline{to_tsvector('foo qux') @@ to_tsquery('foo | bar')} \\
    \sqlinline{expr & expr}
        & boolean AND
        & \sqlinline{to_tsvector('bar baz foo') @@ to_tsquery('foo & bar')} \\
    \sqlinline{! expr}
        & boolean NOT
        & \sqlinline{to_tsvector('bar baz qux') @@ to_tsquery('! foo')} \\
    \sqlinline{string:*}
        & prefix match
        & \sqlinline{to_tsvector('bar foo baz') @@ to_tsquery('fo:*')} \\
    \bottomrule
  \end{tabular}
\end{center}

\sqlinline{anyarray} operators:

\begin{center}
  \begin{tabular}{lll}
    \toprule
    signature & description & example \\
    \midrule
    \sqlinline{anyarray && anyarray}
        & overlap
        & \sqlinline{ARRAY[1, 4, 3] && ARRAY[2, 1]} \\
    \sqlinline{anyarray <@ anyarray}
        & is contained by
        & \sqlinline{ARRAY[2, 2, 7] <@ ARRAY[1, 7, 4, 2, 6]} \\
    \sqlinline{anyarray = anyarray}
        & equal
        & \sqlinline{ARRAY[1.1,2.1,3.1]::int[] = ARRAY[1,2,3]} \\
    \sqlinline{anyarray @> anyarray}
        & contains
        & \sqlinline{ARRAY[1,4,3] @> ARRAY[3,1,3]} \\
    \bottomrule
  \end{tabular}
\end{center}

\sqlinline{jsonb} operators:

\begin{center}
  \begin{tabular}{lll}
    \toprule
    signature & description & example \\
    \midrule
    \sqlinline{jsonb ? text}
        & Is string \sqlinline{text} a top-level key in \sqlinline{jsonb}?
        & \sqlinline{'{"a":1, "b":2}'::jsonb ? 'b'} \\
    \sqlinline{jsonb ?& text[]}
        & Are all strings in \sqlinline{text[]} a top-level key
        & \sqlinline{'{"a":1, "b":2, "c":3}'::jsonb} \\
      & in \sqlinline{jsonb}?
        & \sqlinline{?& ARRAY['b', 'c']} \\
    \sqlinline{jsonb ?| text[]}
        & Are any strings in \sqlinline{text[]} a top-level key
        & \sqlinline{'{"a":1, "b":2}'::jsonb} \\
      & in \sqlinline{jsonb}?
        & \sqlinline{?| ARRAY['b', 'c']} \\
    \sqlinline{jsonb @> jsonb}
        & Does the left \sqlinline{jsonb} contain the
        & \sqlinline{'{"a":1, "b":2}'::jsonb} \\
      & right \sqlinline{jsonb} at the top level?
        & \sqlinline{@> '{"b":2}'::jsonb} \\
    \sqlinline{jsonb @? jsonpath}
        & Does \sqlinline{jsonpath} for \sqlinline{jsonb} return any item?
        & \sqlinline{'{"a":[1,2,3,4]}'::jsonb} \\
      & \sqlinline{WITH w AS (SELECT jsonb_path_query(}
        & \sqlinline{@? '$.a[*] ? (@ > 2)'} \\
      & \sqlinline{    <jsonb>, <jsonpath>))} \\
      & \sqlinline{  SELECT count(*) > 0 FROM w;} \\
    \sqlinline{jsonb @@ jsonpath}
        & Is the first item of \sqlinline{jsonpath} for \sqlinline{jsonb} true?
        & \sqlinline{'{"a":[1,2,3,4]}'::jsonb} \\
      & \sqlinline{SELECT jsonb_path_match(}
        & \sqlinline{@? '$.a[*] > 2'} \\
      & \sqlinline{    <jsonb>, <jsonpath>, '{}', true);} \\
    \bottomrule
  \end{tabular}
\end{center}

\sqlinline{jsonpath} expression operators:

Given \sqlinline{j jsonb := '{"a":[2,4,"b",2,"c",0.5], "d":["ef","gh"]}'},

\begin{center}
  \begin{tabular}{llll}
    \toprule
    operator
        & description
        & example: \sqlinline{jsonb_path_query(j, } $\cdot$ \sqlinline{)}
        & result \\
    \midrule
    \sqlinline{value == value}
        & equality
        & \sqlinline{'$.a[*] ? (@ == 2)'}
        & \sqlinline{2, 2} \\
    \sqlinline{value > value}
        & greater-than
        & \sqlinline{'$.a[*] ? (@ > 1)'}
        & \sqlinline{2, 4, 2} \\
    \sqlinline{value != value}
        & non-equality
        & \sqlinline{'$.a[*] ? (@ != 4)'}
        & \sqlinline{2, 2, 0.5} \\
      && \sqlinline{'$.a[*] ? (@ != 3)'}
        & \sqlinline{2, 4, 2, 0.5} \\
      && \sqlinline{'$.a[*] ? (@ != "b")'}
        & \sqlinline{"c"} \\
      && \sqlinline{'$.e ? (@ != 3)'}
        & none \\
      && \sqlinline{'$.e != 3'}
        & \sqlinline{false} \\
    \sqlinline{! boolean}
        & boolean NOT
        & \sqlinline{'$.a[*] ? (!(@ == 4))'}
        & \sqlinline{2, 2, 0.5} \\
      && \sqlinline{'$.a[*] ? (!(@ == 3))'}
        & \sqlinline{2, 4, 2, 0.5} \\
      && \sqlinline{'$.a[*] ? (!(@ == "b"))'}
        & \sqlinline{"c"} \\
      && \sqlinline{'!($.e == 3)'}
        & \sqlinline{true} \\
    \sqlinline{string like_regex string}
        & regex match
        & \sqlinline{'$.d[*] ? (@ like_regex "[hi]$")'}
        & \sqlinline{"gh"} \\
    \sqlinline{string starts with string}
        & prefix match
        & \sqlinline{'$.d[*] ? (@ starts with "e")'}
        & \sqlinline{"ef"} \\
    \sqlinline{number . ceiling()}
        & math ceiling
        & \sqlinline{'$.a[*] ? (@.ceiling())'}
        & \sqlinline{2, 4, 2, 1} \\
    \sqlinline{value . type()}
        & JSON type
        & \sqlinline{'$.a[*] ? (@.type() == "string")'}
        & \sqlinline{"b", "c"} \\
    \bottomrule
  \end{tabular}
\end{center}
