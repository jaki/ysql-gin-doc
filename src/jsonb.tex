The GIN index key should be paths, like

\begin{oparts}
\item
  \textinline{<JSON>, a, 1}
\item
  \textinline{<JSON>, b, <JSON>, c, d}
\item
  \textinline{<JSON>, b, <JSON>, c, e}
\item
  \textinline{<JSON>, b, <ARRAY>, 1}
\item
  \textinline{<JSON>, b, <ARRAY>, <ARRAY>, 2}
\item
  \textinline{<JSON>, b, <ARRAY>, <ARRAY>, 3}
\item
  \textinline{<JSON>, b, <ARRAY>, 4}
\item
  \textinline{<JSON>, c, f}
\end{oparts}

(This is inspired by \href{%
  https://github.com/cockroachdb/cockroach/blob/master/docs/RFCS/20171020_inverted_indexes.md}{%
  CockroachDB's inverted index RFC}.)

Contains (\sqlinline{@>}) searches can have corresponding scan keys

\begin{oparts}
\item
  \sqlinline{j @> '{"a": 1}'}
  scans for \textinline{<JSON>, a, 1}
\item
  \sqlinline{j @> '{"b": {}}'}
  scans for
  \textinline{<JSON>, b, <JSON>}
\item
  \sqlinline{j @> '{"b": []}'}
  scans for
  \textinline{<JSON>, b, <ARR>}
\item
  \sqlinline{j @> '{"b": [[]]}'}
  scans for
  \textinline{<JSON>, b, <ARR>, <ARR>}
\item
  \sqlinline{j @> '{"b": [4]}'}
  scans for
  \textinline{<JSON>, b, <ARR>, 4}
\end{oparts}

Concerns

\begin{oparts}
\item
  \sqlinline{25.0} and \sqlinline{25} match should equally match--what will the
  number format be like?
\item
  What will the text encoding be like, especially for weird unicode, and how
  does prefix matching work, then?
\item
  How will the end of the JSON document be marked? (Same question for marking
  array end on array GIN index.) \cppinline{kGroupEnd}? But doesn't this mean
  the contents of the JSON GIN key have to be encoded in some way to make
  exclamation marks unambiguous?
\end{oparts}
