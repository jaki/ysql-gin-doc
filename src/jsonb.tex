For \sqlinline{jsonb_ops}, the GIN key is of type \sqlinline{text}.  Like
tsvector, we can encode to DocDB \cppinline{kString}.  For example,
\sqlinline{"\001abc"} can be encoded to \cppinline{"S\x01abc\x00\x00"} for
ascending.  There may be prefix operations on strings using the
\sqlinline{starts with} jsonpath operator, but since \sqlinline{jsonb_ops}
isn't geared towards solving those queries, it should stay largely hash
partitioned.

For \sqlinline{jsonb_path_ops}, the GIN key is of type \sqlinline{int4}.
Internally, it seems to be unsigned 4-byte int, so let's go with that: we can
encode to DocDB \cppinline{kUInt32}.  For example, \sqlinline{2147483648} can
be encoded to \cppinline{"O\x00\x00\x00\x80"} for ascending and
\cppinline{"g\xff\xff\xff\x4f"} for descending.\footnote{see
\cppinline{AppendUInt32ToKey}}  There is no advantage of using range
partitioning since the ints are hashes.

For \sqlinline{jsonb_full_ops}, the GIN key is of type \sqlinline{text}.  Like
before, encode as DocDB \cppinline{kString}.
