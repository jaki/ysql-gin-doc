Here is an example of using a tsvector GIN index. It is inspired by a \href{%
  https://habr.com/en/company/postgrespro/blog/448746/}{%
  habr blog}.  Run on upstream postgres.

\begin{sqlcode}
CREATE TABLE docs (
    doc text,
    ts tsvector GENERATED ALWAYS AS (to_tsvector('simple', doc)) STORED);
INSERT INTO docs (doc) VALUES
  ('Can a sheet slitter slit sheets?'),
  ('How many sheets could a sheet slitter slit?'),
  ('I slit a sheet, a sheet I slit.'),
  ('Upon a slitted sheet I sit.'),
  ('Whoever slit the sheets is a good sheet slitter.'),
  ('I am a sheet slitter.'),
  ('I slit sheets.'),
  ('I am the sleekest sheet slitter that ever slit sheets.'),
  ('She slits the sheet she sits on.');
SELECT * FROM docs; -- what tsvector looks like
\end{sqlcode}

\begin{sqlcode}
CREATE INDEX ON docs USING GIN (ts);
SET enable_seqscan = OFF;
EXPLAIN SELECT * FROM docs
    WHERE ts @@ to_tsquery('simple', 'many'); -- this is index scan
\end{sqlcode}

Example of what can be done, all using the index:

\begin{sqlcode}
SELECT doc FROM docs WHERE ts @@ to_tsquery('simple', 'many & slitter');
SELECT doc FROM docs WHERE ts @@ to_tsquery('simple', 'many | slitter');
SELECT doc FROM docs WHERE ts @@ to_tsquery('simple', 'slit:* & !slit');
SELECT ts_rank(ts, to_tsquery('simple', 'i & sheet:* & slit:*')) as rank, doc
    FROM docs
    WHERE ts @@ to_tsquery('simple', 'i & sheet:* & slit:*')
    ORDER BY rank DESC;
\end{sqlcode}
