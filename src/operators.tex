See the operators in postgresql docs:

\begin{oparts}
\item
  \href{%
    https://www.postgresql.org/docs/current/functions-textsearch.html}{%
    tsvector operators}
\item
  \href{%
    https://www.postgresql.org/docs/current/functions-array.html}{%
    anyarray operators}
\item
  \href{%
    https://www.postgresql.org/docs/current/functions-json.html}{%
    jsonb operators}
\end{oparts}

All the operators boil down to a combination of these primitives:

\begin{oparts}
\item
  $\opget_I(f)$:
  given index $I$ and predicate $f$, return list of tuples
  \[
    T = \braces{(\ik, \pk) \in I : f(\ik)}
  \]
  $f$ should be a simple predicate like
  \begin{oparts}
  \item
    $\ik < c$
  \item
    $\ik = c$
  \item
    $\ik > c$
  \item
    $\ik \geq c$
  \item
    $\ik \leq c$
  \item
    $\ik~\textnormal{starts with}~c$
  \end{oparts}
\item
  $\opand(T_1, T_2)$:
  given list of tuples $T_1$, $T_2$, return
  \[
    T = \braces{(\ik, \pk) \in T_1 \cup T_2 : \pk \in K}
  \]
  where
  \[
    K = T_1[\pk] \cap T_2[\pk]
  \]
\item
  $\opor(T_1, T_2)$:
  given list of tuples $T_1$, $T_2$, return
  \[
    T = T_1 \cup T_2
  \]
\item
  $\oprecheck(T, g)$:
  given list of tuples $T$ and predicate $g$, return list of tuples
  \[
    T' = \braces{(\ik, \pk) \in T : g(\pk)}
  \]
  $g$ may be any predicate like
  \begin{oparts}
  \item
    \sqlinline{pk @@ '$.a.b.c.ceiling() == 5'}
  \item
    \sqlinline{pk @@ '$.d.e.f like_regex "^foo.*bar"'}
  \end{oparts}
\end{oparts}

For example, the \sqlinline{=} operator for arrays makes sure that the indexed
array column equals the query array. This can be done by

\begin{nparts}
\item
  For each distinct element $e$ in the query array $A_q$, get a set of tuples
  $T_e$ where the tuple's row's array contains the element $e$:
  \[
    \forall e \in A_q, T_e = \opget_I(f)
  \]
  where
  \[
    f(\ik) = \parens{\ik \qeq e}
  \]
\item
  Intersect the tuples $T_e$ to filter out any tuples whose row's array doesn't
  contain \textbf{all} of the query array's elements:
  \[
    T = \opand_{e \in A_q}(T_e)
  \]
\item
  Recheck each row to make sure its array has the same number of (distinct)
  elements as the query array:
  \[
    T' = \oprecheck(T, g)
  \]
  where
  \[
    g(\pk) = \parens{\verts{\braces{(i, p) \in T : p = \pk}} \qeq \verts{A_q}}
  \]
\end{nparts}

Recheck is the fallback we can rely on for complicated queries. We can slowly
phase out the need to use recheck as DocDB becomes more capable.

See translations of all operators to primitives in \protect\hyperlink{%
  operators-to-primitives}{%
  the appendix}.
